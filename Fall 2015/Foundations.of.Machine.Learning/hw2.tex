\documentclass[11pt, oneside]{amsart}   	% use "amsart" instead of "article" for AMSLaTeX format
\usepackage{geometry}                		% See geometry.pdf to learn the layout options. There are lots.
\geometry{letterpaper}                   		% ... or a4paper or a5paper or ... 
%\geometry{landscape}                		% Activate for rotated page geometry
%\usepackage[parfill]{parskip}    		% Activate to begin paragraphs with an empty line rather than an indent
\usepackage{graphicx}				% Use pdf, png, jpg, or eps§ with pdflatex; use eps in DVI mode
								% TeX will automatically convert eps --> pdf in pdflatex		
\usepackage{amssymb}
\renewcommand\thesection{\Alph{section}}
%SetFonts

%SetFonts


\title{Homework 2}
\author{Yunlin Zhang}
%\date{}							% Activate to display a given date or no date

\begin{document}
\maketitle
\section{Problem A}
\subsection{VC-dimension of convex combinations}$\\$

$\mathcal{F}=\{\text{sgn}(\sum\limits_{t=1}^T \alpha_t h_t):h_t\in H,\alpha_t\geq0,\sum\limits_{i=1}^T\alpha_t\leq1\}$\\

Let $\text{VCdim(H)}=d$, and the VC-dimension of set of linear threshold functions in $\mathbb{R}^T$ is $T+1$.\\

$\mathcal{F}$ is like a neural network with 1 hidden layer of concept class $H$. Therefore,\\
Let $d=\sum_{t=1}^T \text{VCdim}_t(m)+\text{VCdim}_\text{linear}(m)=dT+T+1<(T+1)(d+1)$ and there are T+1 nodes in this set up.\\

Using theorem 1 from Baum and Haussler$^{[1]}$, where $d=(d+1)(T+1)$, $N=T+1$\\

$\Pi_\mathcal{F}(m)\leq[\frac{(T+1)em}{(d+1)(T+1)}]^{(d+1)(T+1)}<2^m$, the last relationship is based on the assumption of finite VC-dimension for $\mathcal{F}\Rightarrow\text{VCdim}(\mathcal{F})\leq\min\{m:\Pi_\mathcal{F}(m)<2^m\}$\\ 
\indent $\indent\ [\frac{(T+1)em}{(d+1)(T+1)}]^{(d+1)(T+1)}<2^m$\\
\indent $\Leftrightarrow (d+1)(T+1)\log_2[\frac{(T+1)em}{(d+1)(T+1)}]<m$\\
Using the hint from 2014, setting $x=(d+1)(T+1)$ and $y=(T+1)e/[(d+1)(T+1)]$\\
\indent $\Leftrightarrow m=2(d+1)(T+1)\log_2[(T+1)e]\geq 1,\ xy=(T+1)e>4,\ x,y>0,\ \text{and}\ m\geq 1$ and the inequality is satisfied.\\
\indent $\Rightarrow \text{VCdim}(\mathcal{F})\leq m=2(d+1)(T+1)\log_2[(T+1)e]\qed$\\

Reference:  \\
1. E. B. Baum and D. Haussler, What size net gives valid generalization?, Adv. Neural Inform. Process. Systems I, pp. 8190, Morgan Kaufmann,1989.\\
2. Yoav Freund and Robert E. Schapire. A decision-theoretic generalization of on-line learning and an application to boosting. Journal of Computer and System Sciences, 55(1):119-139, 1997. 

\newpage
\section{Problem B}
\subsection{$\{X^+\cup \{x_{m+1}\},X^-\}$ and $\{X^+,X^-\cup \{x_{m+1}\}\}$ are both linearly separable dichotomies by hyperplanes going through the origin (A) iff $\{X^+,X^-\}$ is linearly separable by a hyperplane going through the origin and $x_{m+1}$ (B)}$\\$

First, to show (A)$\Rightarrow$(B)\\
(A)$\Rightarrow\exists\bold{w}_1$ at origin $\ s.t.\ \bold{w}_1\cdot\bold{x}^+>0\ \forall\bold{x}^+\in X^+\cup \{x_{m+1}\},\ \bold{w}_1\cdot \bold{x}^-<0 \ \forall\bold{x}^-\in X^-$\\
\indent and\\
\indent\indent $\exists\bold{w}_2$ at origin $\ s.t.\ \bold{w}_2\cdot\bold{x}^+>0\ \forall\bold{x}^+\in X^+,\ \bold{w}_2\cdot \bold{x}^- <0\ \forall\bold{x}^-\in X^- \cup \{x_{m+1}\}$\\
Define the hyperplane defined by $\bold{w}_1$ as $U_1$ and by $\bold{w}_2$ as $U_2$\\

Consider $f(\alpha)=[\alpha\bold{w}_1+(1-\alpha)\bold{w}_2]\cdot\bold{x}_{m+1},\ \alpha\in[0,1]$\\
$f(\alpha)$ is a linear function of $\alpha$ therefore is continuous in $\alpha\in[0,1]$ and $f(0)<0$, $f(1)>0$, by intermediate value theorem $\exists\alpha'\in(0,1)\ s.t. \ f(\alpha')=0$\\
$\Rightarrow \bold{y}=\alpha'\bold{w}_1+(1-\alpha')\bold{w}_2\ s.t. \ \bold{y}\cdot\bold{x}_{m+1}=0$\\
$\Rightarrow\bold{y}$ is a normal vector that defines a hyperplane $U$ containing $\bold{x}_{m+1}$\\
$\because \alpha',(1-\alpha')>0$\\
$\bold{y}\cdot\bold{x}^+=\alpha'\bold{w}_1\cdot\bold{x}^++(1-\alpha')\bold{w}_2\cdot\bold{x}^+>0$, $\forall\bold{x}^+\in X^+$\\
$\bold{y}\cdot\bold{x}^-=\alpha'\bold{w}_1\cdot\bold{x}^-+(1-\alpha')\bold{w}_2\cdot\bold{x}^-<0$, $\forall\bold{x}^-\in X^-$\\
$\Rightarrow\{X^+,X^-\}$ is linearly separable by the hyperplane defined by $\bold{y}$\\

By construction, since $\alpha'\in(0,1)$, $U$ must be between $U_1$ and $U_2$ \\
By squeeze theorem $U_1$ and $U_2$ go through the origin$\Rightarrow U$ also goes through the origin\\
$\Rightarrow\{X^+,X^-\}$ is a dichotomy that is linearly separable by a hyperplane that goes through the origin and $\bold{x}_{m+1}$. This completes the proof for (A)$\Rightarrow$(B)\\

Now, to show (B)$\Rightarrow$(A)\\
Consider the family of planes defined by the set of vectors $\{\bold{y}(\epsilon)=(1-\epsilon)\bold{w}+\epsilon\bold{x}_{m+1}:\epsilon\in (0,1]\}$\\ 
Using this definition, \\
\indent$\bold{y}\cdot\bold{x}_{m+1}=\epsilon||\bold{x}_{m+1}||^2>0$\\
We'd want to find an $\epsilon$ where $\bold{y}\cdot\bold{x}>0,\ \forall \bold{x}\in X^+$ and $\bold{y}\cdot\bold{x}<0,\ \forall \bold{x}\in X^-$\\
Denote any point in $X^+$ as $\bold{x}^+$ and in $X^-$ as $\bold{x}^-$:\\
For points in $X^+$ to be correctly classified,\\
\indent$\bold{y}\cdot\bold{x}^+=(1-\epsilon)\bold{w}\cdot\bold{x}^++\epsilon\bold{x}_{m+1}\cdot\bold{x}^+>0$\\
\indent$\Leftrightarrow(1-\epsilon)\bold{w}\cdot\bold{x}^+>|\epsilon\bold{x}_{m+1}\cdot\bold{x}^+|$\\
\indent$\Leftrightarrow(1-\epsilon)/\epsilon>|\bold{x}_{m+1}\cdot\bold{x}^+|/\bold{w}\cdot\bold{x}^+$\hfill$\because\epsilon>0,\ \bold{w}\cdot\bold{x}^+>0$\\
\indent$\Leftrightarrow 1/\epsilon>|\bold{x}_{m+1}\cdot\bold{x}^+|/\bold{w}\cdot\bold{x}^++1$\\
\indent$\Leftrightarrow \epsilon<[|\bold{x}_{m+1}\cdot\bold{x}^+|/\bold{w}\cdot\bold{x}^++1]^{-1}=\delta_1>0$\\
\indent Choose $\delta_1'=\min_{\bold{x}^+}\delta_1$\\
\indent$\therefore$ if $\epsilon<\delta_1'$ then all the points in $X^+$ will still be correctly classified\\
Similarly, for points in $X^-$ to be correctly classified,\\
\indent$\bold{y}\cdot\bold{x}^-=(1-\epsilon)\bold{w}\cdot\bold{x}^-+\epsilon\bold{x}_{m+1}\cdot\bold{x}^-<0$\\
\indent$\Leftrightarrow (1-\epsilon)\bold{w}\cdot\bold{x}^-<-|\epsilon\bold{x}_{m+1}\cdot\bold{x}|$\\
\indent$\Leftrightarrow (1-\epsilon)/\epsilon>-|\epsilon\bold{x}_{m+1}\cdot\bold{x}|/\bold{w}\cdot\bold{x}^-$\hfill$\because \bold{w}\cdot\bold{x}^-<0$\\
\indent$\Leftrightarrow\epsilon< [-|\bold{x}_{m+1}\cdot\bold{x}^+|/\bold{w}\cdot\bold{x}^-+1]^{-1}=\delta_2>0$\\
\indent Choose $\delta_2'=\min_{\bold{x}^-}\delta_2$\\
\indent$\therefore$ if $\epsilon<\delta_2'$ then all the points in $X^-$ will still be correctly classified\\
So if we choose some $\epsilon'<\min(\delta_1',\delta_2')\Rightarrow \\
\indent\indent\bold{y}(\epsilon')\cdot\bold{x}>0\ \forall\bold{x}\in X^+\\
\indent\indent\bold{y}(\epsilon')\cdot\bold{x}<0\ \forall\bold{x}\in X^-\\
\indent\indent\bold{y}(\epsilon')\cdot\bold{x}_{m+1}>0
$\\
Therefore the dichotomy $\{X^+\cup \{\bold{x}_{m+1}\}, X^-\}$ is linearly separable by $\bold{y}(\epsilon')$\\

Similar, to show that the dichotomy $\{X^+,X^-\cup\{\bold{x}_{m+1}\}$ is linearly separable, we use the family of planes defined by vectors in $\{\bold{z}(\epsilon)=(1-\epsilon)\bold{w}-\epsilon\bold{x}_{m+1}\:\epsilon\in (0,1]\}$\\
\indent$\Rightarrow\bold{z}\cdot\bold{x}_{m+1}=-\epsilon||\bold{x}_{m+1}||^2<0$\\
The conditions:\\
\indent$\bold{z}\cdot\bold{x}^+=(1-\epsilon)\bold{w}\cdot\bold{x}^+-\epsilon\bold{x}_{m+1}\cdot\bold{x}^+>0$\\
\indent and\\
\indent$\bold{z}\cdot\bold{x}^-=(1-\epsilon)\bold{w}\cdot\bold{x}^--\epsilon\bold{x}_{m+1}\cdot\bold{x}^-<0$\\
simplify to the same forms:\\
\indent$(1-\epsilon)\bold{w}\cdot\bold{x}^+>|\epsilon\bold{x}_{m+1}\cdot\bold{x}^+|$\\
\indent and \\
\indent$(1-\epsilon)\bold{w}\cdot\bold{x}^-<-|\epsilon\bold{x}_{m+1}\cdot\bold{x}|$\\
So we can choose the same $\epsilon'$ as defined in the previous part and the following conditions will hold:\\
\indent\indent$\bold{z}(\epsilon')\cdot\bold{x}>0\ \forall\bold{x}\in X^+\\
\indent\indent\bold{z}(\epsilon')\cdot\bold{x}<0\ \forall\bold{x}\in X^-\\
\indent\indent\bold{z}(\epsilon')\cdot\bold{x}_{m+1}<0\\$
Therefore the dichotomy $\{X^+\cup \{\bold{x}_{m+1}\}, X^-\}$ is linearly separable by $\bold{z}(\epsilon')$, and this completes the proof for (B)$\Rightarrow$(A)$\qed$\\

\subsection{Let X=$\{\bold{x}_1,...,\bold{x}_m\}\subset\mathbb{R}^d\ s.t.$ any $d$-element subset of $X$ is linearly independent. Then the number of linearly separable labelings of $X$ is $C(m,d)=2\sum_{k=0}^{d-1}\binom{m-1}{k}$ }$\\$
The proof is by complete induction on $m+d$ and follow very closely to that of Sauer's lemma.\\

Base case: \\
\indent For any $m$, $d=1$\\
$C(m,1) = 2\binom{m-1}{0} = 2$, which is to say the data points in $\mathbb{R}$ are degenerate and can only be labeled all +1 or all -1\\
\indent For and $d$, $m=1$\\
$C(1,d) = 2\binom{0}{0} = 2$, for any dimension d, if there is only 1 point there can only be 2 ways of labeling it.\\

Inductive case: Assume all $m'+d'<m+d$ true. 

Let the set of all linearly separable labelings of $X=\{\bold{x}_1,...,\bold{x}_{m-1}\}$ be $G$ and $|G|=C(m,d)$. Let $T=\{\bold{x}_1,...,\bold{x}_{m-1}\}$ and denote the set of linearly separable labelings of this as $G_T$. 

Construct set $G_2=\{g'\subseteq T:(g'\in G)\land(g'\cup\{\bold{x}_m\}\in G)\}$. This is the set where each labeling is in the overall set $G$ but not in $G_T$. Specifically, if $U\subseteq T$ where $|U|=d-1$ such that $U\cup\{\bold{x}_m\}$ admits only 1 possible labeling for $\bold{x}_m$ then this particular $g'$ is in $G_T$, otherwise if both labeling are possible, then $g'$ will be in both $G_T$ and $G_2$. Using this definition, $|G|=|G_T|+|G_2|$.

Now we need to find $|G_T|$ and $|G_2|$. For any labeling of $X=\{\bold{x}_1,...,\bold{x}_m\}$, the labeling for any $m-1$ elements must be in $G_T$, and the maximum number of linearly separable points in $G_T$ is still $d\Rightarrow|G_T|=C(m-1,d)$

And for any labeling of $U$, a $d-1$ element subset of $X$, to be in $G_2$, both possible labeling for $\bold{x}_m\in T\cup\{ \bold{x}_m\}$ must be in $G\Rightarrow$ the maximum number of linearly separable points in $G_2$ must be $d-1\Rightarrow |G_2|=C(m-1,d-1)$\\
Using the following properties:\\
\indent $\binom{m}{k}=\binom{m-1}{k}+\binom{m-1}{k-1}$\\
Using the induction hypothesis:\\
$C(m,d)=C(m-1,d)+C(m-1,d-1)=2\sum_{k=0}^{d-1}\binom{m-2}{k}+2\sum_{k=0}^{d-2}\binom{m-2}{k}$\\
\indent\indent\indent $=2\sum_{k=0}^{d-1}[\binom{m-2}{k}+\binom{m-2}{k-1}]$\\
\indent\indent\indent $=2\sum_{k=0}^{d-1}\binom{m-1}{k}$\\
which completes the proof $\qed$\\

\subsection{Growth function of linear combination of linearly independent variable}$\\$
\indent $\mathcal{F}=\{x\mapsto\text{sgn}(\sum\limits_{k=1}^p a_k f_k(x)):a_1,...,a_p\in\mathbb{R}\}$\\
Using results from the previous part, let $U=\{\Phi(x_1),...,\Phi(x_m)\}$ where every $p$-subset is linearly independent. Then the total number of linearly separable labeling of $U$ is $2\sum_{i=1}^{p-1}\binom{m-1}{i}\Rightarrow\Pi_\mathcal{F}(m)\leq 2\sum_{i=1}^{p-1}\binom{m-1}{i}$. We need to show that this relationship is strictly equal\\

By definition, since any $p$-element subset of $U$ is linearly independent, for any such subset, $\exists \bold{a} \in\mathbb{R}^p$, choose this $\bold{a}$ to plug into the definition of $\mathcal{F}$ and we will be able to attain any combination labeling of $U$, and therefore all labelings of $U$ are possible using $\mathcal{F}$ and the equality is satisfied.$\qed$


\newpage
\section{Problem C}

\newpage
\section{Problem D}
\subsection{Show $K(\bold{x},\bold{y})=\sum_{i=1}^N\cos^n(x_i^2-y_i^2)\ \ \forall(\bold{x},\bold{y})\in \mathbb{R}^N\times\mathbb{R}^N$ is PDS}$\\$
Using $\cos(\alpha-\beta)=\cos\alpha\cos\beta+\sin\alpha\sin\beta=
\left[
\begin{array}{c}
\cos\alpha\\
\sin\alpha\\
\end{array}
\right]
\cdot
\left[
\begin{array}{c}
\cos\beta\\
\sin\beta\\
\end{array}
\right]
$\\
Let $\Phi_i(\bold{x})=
\left[
\begin{array}{c}
\cos x_i^2\\
\sin x_i^2\\
\end{array}
\right]$ and $K_i(\bold{x},\bold{y})=\Phi_i(\bold{x})\cdot\Phi_i(\bold{y})$ is PDS\\
$K(\bold{x},\bold{y})=\sum_{i=1}^N [K_i(\bold{x},\bold{y})]^n\Rightarrow K(\bold{x},\bold{y})$ is PDS by closure of PDS kernels.$\qed$
$\\$
\subsection{Show $K(\bold{x},\bold{y})=\exp(-||\bold{x}-\bold{y}||/\sigma)\ \ \forall(\bold{x},\bold{y})\in \mathbb{R}^N\times\mathbb{R}^N$ is PDS}$\\$
Consider if $\sum_{i=1}^mc_i=0$,\\
$\sum_{i,j=1}^m c_i c_j ||\bold{x}_i-\bold{x}_j||=\sum_{i,j}^m c_i c_j \sqrt{(\bold{x}_i-\bold{x}_j)\cdot(\bold{x}_i-\bold{x}_j)}$\\
Case 1: $(\bold{x}_i-\bold{x}_j)\cdot(\bold{x}_i-\bold{x}_j)\leq1$\\
$\sum_{i,j=1}^m c_i c_j ||\bold{x}_i-\bold{x}_j||=\sum_{i,j} c_i c_j \sqrt{(\bold{x}_i-\bold{x}_j)\cdot(\bold{x}_i-\bold{x}_j)}$\\
\indent$\leq\sum_{i,j}c_i c_j=\sum_i c_i\sum_j c_j=0$\\
Case 2: $(\bold{x}_i-\bold{x}_j)\cdot(\bold{x}_i-\bold{x}_j)>1\Rightarrow\sqrt{(\bold{x}_i-\bold{x}_j)\cdot(\bold{x}_i-\bold{x}_j)}\leq(\bold{x}_i-\bold{x}_j)\cdot(\bold{x}_i-\bold{x}_j)$\\
$\sum_{i,j=1}^m c_i c_j ||\bold{x}_i-\bold{x}_j||=\sum_{i,j} c_i c_j \sqrt{(\bold{x}_i-\bold{x}_j)\cdot(\bold{x}_i-\bold{x}_j)}$\\
\indent$\leq\sum_{i,j} c_i c_j (\bold{x}_i-\bold{x}_j)\cdot(\bold{x}_i-\bold{x}_j)$\\
\indent$=\sum_{i,j} c_i c_j (||\bold{x}_i||^2+||\bold{x}_j||^2-2\bold{x}_i\cdot\bold{x}_j)$\\
\indent$=\sum_{i,j} c_i c_j (||\bold{x}_i||^2+||\bold{x}_j||^2)-2\sum_{i,j} c_i c_j (\bold{x}_i\cdot\bold{x}_j)$\\
\indent$=\sum_{i,j} c_i c_j (||\bold{x}_i||^2+||\bold{x}_j||^2)-2\sum_{i} c_i \bold{x}_i\cdot\sum_{j}c_j \bold{x}_j$\\
\indent$=\sum_{i,j} c_i c_j (||\bold{x}_i||^2+||\bold{x}_j||^2)-2||\sum_{i} c_i \bold{x}_i||^2$\\
\indent$\leq\sum_{i,j} c_i c_j (||\bold{x}_i||^2+||\bold{x}_j||^2)$\\
\indent$=\sum_i c_i\sum_j c_j||\bold{x}_j||^2+\sum_j c_j \sum_i c_i||\bold{x}_i||^2=0$\hfill$\because\sum_i c_i=0$\\
$\Rightarrow\sum_{i,j=1}^m c_i c_j ||\bold{x}_i-\bold{x}_j||\leq0,\ \forall(\bold{x}_i,\bold{x}_j)\in \mathbb{R}^N\times\mathbb{R}^N$\\
$\Rightarrow\ K_0(\bold{x},\bold{y})=||\bold{x}-\bold{y}||$ is an NDS kernel as defined in textbook.\\
$\Rightarrow\ K(\bold{x},\bold{y})=\exp(-K_0/\sigma)$, and $1/\sigma>0\Rightarrow K(\bold{x},\bold{y})$ is PDS.$\qed$\\









\end{document}  