\documentclass[11pt, oneside]{article}   	% use "amsart" instead of "article" for AMSLaTeX format
\usepackage{geometry}                		% See geometry.pdf to learn the layout options. There are lots.
\geometry{letterpaper}                   		% ... or a4paper or a5paper or ... 
%\geometry{landscape}                		% Activate for rotated page geometry
%\usepackage[parfill]{parskip}    		% Activate to begin paragraphs with an empty line rather than an indent
\usepackage{graphicx}				% Use pdf, png, jpg, or eps§ with pdflatex; use eps in DVI mode
								% TeX will automatically convert eps --> pdf in pdflatex		
\usepackage{amssymb}
\usepackage{setspace}
%SetFonts

%SetFonts
\onehalfspacing 

\title{Homework 2}
\author{Yunlin Zhang}
\date{2015/10/09}							% Activate to display a given date or no date

\begin{document}
\maketitle
\section{Problem 1}
\subsection{}
For OLS, we can derive the following properties:\\
$Cov[X,Y]=\hat{\beta_1}/\hat{\beta_1}Cov[X,Y]$\\
\indent$=Cov[\hat{\beta_1}X,Y]/\hat{beta_1}$\\
\indent$=Cov[\hat{\beta_1}X+\hat{\beta_0},Y]/\hat{\beta_1}$\\
\indent$=Cov[\hat{Y},Y]/\hat{\beta_1}$\\
$\sigma[X]=\hat{\beta_1}/\hat{\beta_1}\sigma[X]$\\
\indent$=\sigma[\hat{\beta_1}X]/\hat{\beta_1}$\\
\indent$=\sigma[\hat{\beta_1}X+\hat{\beta_0}]/\hat{\beta_1}$\\
\indent$=\sigma[\hat{Y}]/\hat{\beta_1}$\\
$\Rightarrow\rho_{xy}=Cov[X,Y]/\sigma[X]\sigma[Y]=Cov[\hat{Y},Y]/\sigma[\hat{Y}]\sigma[Y]$\\
\indent$=\sum(\hat{y_i}-\overline{y})(y_i-\overline{y})/\sqrt{\sum(\hat{y_i}-\overline{y})^2}\sqrt{\sum(y_i-\overline{y})^2}$\\
\indent$=\sum(\hat{y_i}-\overline{y})(y_i-\hat{y_i}+\hat{y_i}-\overline{y})/\sqrt{\sum(\hat{y_i}-\overline{y})^2}\sqrt{\sum(y_i-\overline{y})^2}$\\
\indent$=\sum[(\hat{y_i}-\overline{y})(y_i-\hat{y_i})+(\hat{y_i}-\overline{y})(\hat{y_i}-\overline{y})]/\sqrt{\sum(\hat{y_i}-\overline{y})^2}\sqrt{\sum(y_i-\overline{y})^2}$\\
\indent$=\sum[(\hat{y_i}-\overline{y})\hat{u_i}+(\hat{y_i}-\overline{y})^2]/\sqrt{\sum(\hat{y_i}-\overline{y})^2}\sqrt{\sum(y_i-\overline{y})^2}$\\
\indent$=\sum[(\hat{y_i}-\overline{y})^2]/\sqrt{\sum(\hat{y_i}-\overline{y})^2}\sqrt{\sum(y_i-\overline{y})^2}$\hfill$\because\sum(\hat{y_i}-\overline{y})\hat{u_i}=0$\\
\indent$=\sqrt{\sum[(\hat{y_i}-\overline{y})^2]}/\sqrt{\sum(y_i-\overline{y})^2}$\\
\indent$=\sqrt{SSE/SST}$\\
\indent$=\sqrt{R^2}$\\
$\Rightarrow\rho_{xy}^2=R^2$ for single variate OLS\\

\subsection{}
In general, $R^2\neq\rho_{xy}^2$ in the multivariate case. Observe that in the proof for the single variable case, a transformation is made from $X$ to $\hat{Y}$. However, in the multivariable case, the covariance terms would not reduce to 0 after making the transformation, and therefore the two values would not converge.\\

\section{Problem 2}
\subsection{}
For a general linear model of the form $y_i=\sum_{j=0}^k\beta_j x_{ij} +u_i$\\
If MLR.1-MLR.6 are satisfied\\
\indent $\Rightarrow \hat{\beta_j}\sim \mathcal{N}(\beta_j, \sigma^2/[SST_j(1-R_j^2)])$\\
\indent For simplicity let $SST'_j=SST_j(1-R_j^2)$\\
\indent\indent $se[\hat{\beta_j}]=\sqrt{s^2/SST'_j}$\\
\indent\indent $s^2=\frac{1}{n-k-1}\sum \hat{u}_i^2$\\
\indent By MLR.6 $\hat{u}\sim\mathcal{N}(0,\sigma^2)\Rightarrow \hat{u}/\sigma\sim\mathcal{N}(0,1)\Rightarrow \sum \hat{u_i}^2\sim\sigma^2\mathcal{X}^2$\\
\indent$\Rightarrow s^2(n-k-1)/\sigma^2\sim\mathcal{X}^2_{n-k-1}$\\
$\Rightarrow(\hat{\beta_j}-\beta_j)/se(\hat{\beta_j})=(\hat{\beta_j}-\beta_j)/\sqrt{s^2/SST'_j}$\\
\indent$=(\hat{\beta_j}-\beta_j)/\sqrt{\sigma^2/SST'_j}/\sqrt{s^2/\sigma^2}$\hfill$\because$ multiplying by 1\\
\indent$=(\hat{\beta_j}-\beta_j)/\sqrt{\sigma^2/SST'_j}/\sqrt{s^2(n-k-1)/[\sigma^2(n-k-1)]}$\hfill$\because$multiplying by 1\\
\indent$\sim\mathcal{N}(0,1)/\sqrt{\mathcal{X}^2_{n-k-1}/(n-k-1)}$\\
\indent$\sim t_{n-k-1}$\\


%\indent $\Rightarrow se[\hat{\beta_j}]=\sqrt{\sigma^2/[SST_j(1-R_j^2)]}\sqrt{\mathcal{X}_n^2/(n-k-1)}$\\
%\indent Let $Z=(\hat{\beta_j}-\beta_j)/\sqrt{\sigma^2/[SST_j(1-R_j^2)]}\sim\mathcal{N}(0,1)$\\
%\indent $\Rightarrow (\hat{\beta_j}-\beta_j)/se(\hat{\beta_j})=Z/\sqrt{X_n^2/(n-k-1)}\sim t_{n-k-1}$\\

\subsection{}
Based on results from previous problem, $s^2(df)/\sigma^2\sim\mathcal{X}_{df}^2$\\
Also in general, $SSR=(df)s^2\sim\sigma^2\mathcal{X}_{df}^2$\\
Since unrestricted model has $n-k-1$ degrees of freedom$\Rightarrow$ restricted model has $n-k-1-q$ degrees of freedom\\
This implied:
\indent $({SSR_r}-SSR_{ur})/q\sim\sigma^2(\mathcal{X}_{n-k-1-q}^2-\mathcal{X}_{n-k-1}^2)/q$\\
\indent\indent$\sim\sigma^2(\mathcal{X}_q^2/q)$\\
\indent And\\
\indent$SSR_{ur}/(n-k-1)\sim\sigma^2\mathcal{X}^2_{n-k-q}$\\
\indent The $\sigma^2$ in top and bottom will cancel\\
\indent$\Rightarrow[(SSR_{ur}-SSR_r)/q]/[SSR_{ur}/(n-k-1)]\sim(\mathcal{X}_q^2/q)/(\mathcal{X}_{n-k-1}^2/(n-k-1))$\\
\indent$\sim F_{q,n-k-1}$\\

\section{Problem 3}
$\tilde{\beta_1}=\sum\hat{r}_{i1}y_i/\sum\hat{r}_{i1}^2$
Plugging in $y_i=\beta_0+\beta_1x_{i1}+\beta_2x_{i2}+\beta_3x_{i3}+u_i$\\
Working with the numerator:\\
$\sum\hat{r}_{i1}y_i=\sum\hat{r}_{i1}(\beta_0+\beta_1x_{i1}+\beta_2x_{i2}+\beta_3x_{i3}+u_i)$\\
But $\beta_0+\beta_1x_{i1}+\beta_2x_{i2}=\beta_1\hat{r}_{i1}$\\
\indent$\Rightarrow \sum\hat{r}_{i1}y_i=\beta_1\sum\hat{r}_{i1}^2 + \beta_3\sum\hat{r}_{i1}x_{i3}$\\
\indent$\Rightarrow E[\tilde{\beta}_1=E[(\beta_1\sum\hat{r}_{i1}^2 + \beta_3\sum\hat{r}_{i1}x_{i3})/\sum\hat{r}_{i1}^2]$\\
\indent\indent$=\beta_1 + \beta_3\ \sum\hat{r}_{i1}x_{i3}/\sum\hat{r}_{i1}^2$\\

\section{Problem 4}
$\log{wage}=\beta_0+\beta_1educ+\beta_2exper+\beta_3tenure+u$\\
\subsection{}
Another year of general work experience has as much effect on $\log{wage}$ as another year of tenure with current employer\\
\indent$H_0: \hat{\beta_2}=\hat{\beta_3}$\\
\indent However, this can also be written as:\\
\indent$H_0: \theta=\hat{\beta_2}-\hat{\beta_3}=0$\\
\indent And the model equation will be transformed to:\\
\indent$\log(wage)=\beta_0+\beta_1educ+\theta exper+\beta_2(exper+tenure)$\\
\subsection{}
Calculated t-statistic and p-value for $\theta$: \\
\indent$t_\theta=0.412$\\
\indent$p_\theta=0.680$\\
Based on the t-test, $\theta$ is only non-zero with a 68\% significance. This means that we cannot reject the null hypothesis that $\theta$ is 0, and therefore we cannot reject that the effect of tenure at current position and experience at previous position are the same on current (log)wages.

\section{Problem 5}
$\log(psoda)=\beta_0+\beta_1prpblck+\beta_2\log(income)+\beta_3prppov+u$\\
\subsection{}
Data is cleaned by removing empty entries and 0 entries for income and psoda.
Calculated t-statistics and p-value for $\beta_1$:\\
\indent$t_{\beta_1}=2.373$\\
\indent$p_{\beta_1}=0.018=1.8\%$\\
Based on the t-test, the null hypothesis would be rejected at 5\% significance level, while it would not be rejected at 1\% significance level.
\subsection{}
Calculation is carried out in Excel
$Corr[\log(income),prppov]=-0.840$\\
These two variables are pretty highly correlated, which is also visible based on a scatter plot between them.\\
p-value for prppov = 0.0044\\ and \\
p-value for log(income) = 4.8e-7\\
With very high confidence we can reject the null hypothesis, therefore these two variables are significant in the regression model.
\subsection{}
p-value for log(hseval) = 2.67e-11\\
With very high confidence we can reject the null hypothesis, and therefore this parameter is significant.
\subsection{}
p-value for log(income) became 0.159\\ and \\
p-value for prppov became 0.699\\
Therefore individually, the became non-significant.

For testing join significance:\\
\indent $H_0:\beta_2=\beta_3=0$\\
Running F test by restricting these two variables:
$SSR_r=0.450982$\\
From part iii,
$SSR_{ur}=0.0443098$\\
Running an F-test:
$q = 2$, $n-k-1=401-4-1=396$\\
$F=[(SSR_r-SSR_{ur})/q]/[SSR_{ur}/(n-k-1)]=1857$\\
At 95\% confidence/5\% significance, $F_{2,396}=3$ therefore we reject the null hypothesis, and these variables are significant jointly.

\subsection{}
Based on the tests in the previous sections, the most reliable model is from part iii, where $\log(psoda)=\beta_0+\beta_1prpblck+\beta_2\log(income)+\beta_3prppov+\beta_4\log(hseval)+u$


\section{Problem 6}
The second model is preferable since it has a higher $\overline{R}^2$ value than that of the other two models. Without seeing the data, basing just on the three models, it looks like a linear response on totemp is too strong and had to be corrected by a smaller quadratic term, which makes the log model work better in the regime that the sampled data is in.

\end{document}  